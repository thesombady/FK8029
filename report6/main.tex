\documentclass[a4paper]{article}

% --- Packages ---

\usepackage{a4wide}
\usepackage[utf8]{inputenc}
\usepackage{amsmath}
\usepackage{mathtools}
\usepackage{amssymb}
\usepackage[english]{babel}
\usepackage{mdframed}
\usepackage{systeme,}
\usepackage{lipsum}
\usepackage{relsize}
\usepackage{caption}
\usepackage{tikz}
\usepackage{tikz-3dplot}
\usetikzlibrary{shapes.geometric}
\usepackage{pgfplots}
\usepackage{pgfplotstable}
\pgfplotsset{compat=newest}%1.7}
\usepackage{harpoon}%
\usepackage{graphicx}
\usepackage{wrapfig}
\usepackage{subcaption}
\usepackage{authblk}
\usepackage{float}
\usepackage{listings}
\usepackage{xcolor}
\usepackage{chngcntr}
\usepackage{amsthm}
\usepackage{comment}
\usepackage{commath}
\usepackage{hyperref}%Might remove, adds link to each reference
\usepackage{url}
\usepackage{calligra}
\usepackage{pgf}

% --- Bibtex ---
% To run our bibliography, we need to compile the ref document
% `biber main` or `biber ref` in the terminal
% We can compile the document with `pdflatex main` or `latex main`

\usepackage{csquotes}
\usepackage[
    %backend=biber,
    backend = biber,
    style=phys,
    sorting=ynt,
]{biblatex}

%\addbibresource{ref.bib}


% --- Commands --- 

\newcommand{\w}{\omega}
\newcommand{\trace}{\text{Tr}}
\newcommand{\grad}{\mathbf{\nabla}}
%\newcommand{\crr}{\mathfrak{r}}
\newcommand{\laplace}{\nabla^2}
\newcommand{\newparagraph}{\vspace{.5cm}\noindent}

% --- Math character commands ---

\newcommand{\curl}[1]{\mathbf{\nabla}\times \mathbf{#1}}
\newcommand{\dive}[1]{\mathbf{\nabla}\cdot \mathbf{#1}}
\newcommand{\res}[2]{\text{Res}(#1,#2)}
\newcommand{\fpartial}[2]{\frac{\partial #1}{\partial #2}}
\newcommand{\rot}[3]{\begin{vmatrix}\hat{x}&\hat{y}&\hat{z}\\\partial_x&\partial_y&\partial_z\\#1&#2&#3 \end{vmatrix}}
\newcommand{\average}[1]{\langle #1 \rangle}
\newcommand{\ket}[1]{|#1\rangle}
\newcommand{\bra}[1]{\langle #1|}


%  --- Special character commands ---

\DeclareMathAlphabet{\mathcalligra}{T1}{calligra}{m}{n}
\DeclareFontShape{T1}{calligra}{m}{n}{<->s*[2.2]callig15}{}
\newcommand{\crr}{\mathcalligra{r}\,}
\newcommand{\boldscriptr}{\pmb{\mathcalligra{r}}\,}


\title{INPUT TITLE HERE}
\author{Author : Andreas Evensen}
\date{Date: \today}

% --- Code ---

\definecolor{codegreen}{rgb}{0,0.6,0}
\definecolor{codegray}{rgb}{0.5,0.5,0.5}
\definecolor{codepurple}{rgb}{0.58,0,0.82}
\definecolor{backcolour}{rgb}{0.95,0.95,0.92}

\lstdefinestyle{mystyle}{
    backgroundcolor=\color{backcolour},   
    commentstyle=\color{codegreen},
    keywordstyle=\color{magenta},
    numberstyle=\tiny\color{codegray},
    stringstyle=\color{codepurple},
    basicstyle=\ttfamily\footnotesize,
    breakatwhitespace=false,         
    breaklines=true,                 
    captionpos=b,                    
    keepspaces=true,                 
    numbers=left,                    
    numbersep=5pt,                  
    showspaces=false,                
    showstringspaces=false,
    showtabs=false,                  
    tabsize=2
}

\lstset{style=mystyle}

\begin{document}

\begin{titlepage}
    \begin{center}
        \vspace*{1cm}

        \Huge
        \textbf{Hydrogen atom}

        \vspace{0.5cm}
        \LARGE
        FK8029 - Computational physics

        \vspace{1.5cm}

        \textbf{Andreas Evensen}

        \vfill

        %\includegraphics[width=0.4\textwidth]{UiO_Segl_pantone.eps}

        \Large
        Department of Physics\\
        Stockholm University\\
        Sweden\\
        \today
    \end{center}
\end{titlepage}

\section{Introduction}
In this report, we want to solve the Bohr Hydrogen atom, where we introduce the fixed orbitals $l$.
We accomplish this task by introducing a set of basis-functions, B-splines, which then means that we can write the wave-function as a linear combination of the basis-functions.
Using the B-spline basis, we solve for a variety of states, such as: $1s$, $2s$, and $2p$.
The results agree with tabulated data.


\tableofcontents
\newpage

\section{Theory \& Method}
Hydrogen is the simplest of all atoms, given that it only has one electron orbiting the nucleus.
We will therefore aim to solve the wave-function for electron orbiting the nucleus. Hence, we aim to solve the following equation:
\begin{align}
    \hat{H}\psi(r, \theta, \phi) &= E\psi(r, \theta, \phi).\label{eq: eigen-eq}
\end{align}We state that our wave-function $\psi(r, \theta, \phi)$, can be decomposed into a radial part and an angular part, which is commonly known as the spherical harmonics:
\begin{align*}
    \psi(r, \theta, \phi) = R_{nl}(r)Y_{nl}(\theta, \phi).
\end{align*}Defining a function $P_{nl}(r) = R_{nl}(r)\cdot r$, and expressing $R_{nl}(r)$ in a B-spline basis yields the following expression:
\begin{align}
    \hat{H}\left(\sum_i c_i B_i^k(r)\right) &= E\left(\sum_i c_i B_i^k(r)\right),\\
    \hat{H} &= \hbar^2\left(-\frac{1}{2m_e}\frac{d^2}{dr^2} + \frac{l(l + 1)}{2m_e}\frac{1}{r^2} - \frac{Ze^2}{4\pi\epsilon\hbar^2}\frac{1}{r}\right),\label{eq: hamiltonian}
\end{align}
We multiply from the left with $B_j^k(r)$ and integrate to get the following expression
\begin{align}
    \underbrace{\int\left(\left(\sum_j c_j B_j^k(r)\right)\hat{H}\left(\sum_i c_i B_i^k(r)\right)\right)dr}_{\mathbf{H}\mathbf{c}} &= E\underbrace{\int\left(\left(\sum_i c_j B_j^k(r)\right)\left(\sum_i c_i B_i^k(r)\right)\right)dr}_{\mathbf{B}\mathbf{c}}.\label{eq: gen eq}
\end{align}Hence, eq \eqref{eq: eigen-eq} has been transformed to a generalized eigen-value equation.
To transform back, we simply multiply from the left with $\mathbf{B}^{-1}$.
Hence, this can be solved either with a direct method for an eigen-value equation, or with a generalized eigen-value equation solver. In this report, the latter was used.

\newparagraph
To find the matrix-elements of $\mathbf{H}$ and $\mathbf{B}$ we define the following expressions:
\begin{align}
    \mathbf{B}_{i, j} &= \int_{t_{first}}^{t_{last}}dr\left(B_{j}^k(r)B_i^k(r)\right),\label{eq: B}\\
    \mathbf{H}_{i, j} &= \int_{t_{first}}^{t_{last}}dr\left(B_{j}^k(r)\hat{H}B_i^k(r)\right).\nonumber
\end{align}Either $\hat{H}$ is defined as per eq \eqref{eq: hamiltonian}, or we use integration by parts to obtain the following expression for $\mathbf{H}_{i,j}$:
\begin{align}
    \mathbf{H}_{i,j} = \frac{\hbar^2}{2m_e}\int_{t_{first}}^{t_{last}}dr\left(\frac{dB_j^k}{dx}(r)\frac{dB_i^k}{dx}(r)\right) + \int_{t_{first}}^{t_{last}}dr\left(B_j^k(r)\left(\frac{\hbar^2l(l + 1)}{2m_er^2} - \frac{Ze^2}{4\pi\epsilon_0r}\right)B_i^k(r)\right),\label{eq: H}
\end{align}and this is the expression used to obtain the result shown in section \ref{sec: result}.
The integrals in eq \eqref{eq: B} -- \eqref{eq: H}, can be evaluated in a variety of ways, Riemann integration, trapezoidal methods and so on, but in this report, Gaussian Quadrature integration was chosen.
Gaussian quadrature is a good choice for integration with polynomials as it relies on Legendre polynomials to compute the integral with weighting at certain points.
Gaussian quadrature, in our domain, is defined as follows:
\begin{align}
    \int_{t_n}^{t_{n+ 1}}dr\left(q_1(r)f(r)q_2(r)\right) \approx \sum_{m = 1}^k w_m q_1(r_m)f(r_m)q_2(r_m),\label{eq: gauss-quad}
\end{align}where $w_m$ are the weights for a $k$ or Legendre polynomial and $r_m$ are the corresponding roots, also known as abscissas, of the same polynomial.
So if we have $n$ knots, including left and right ghost-points, then our matrix equation is of size $n - k$, where $k$ is the spline order: e.g. $k = 4$ is cubic B-splines, which spans all the B-splines.
Furthermore, there exists boundary conditions on the left, and right, being that $P_{nl}(r)$ vanishes at $r\to 0$ and that all states bound states goes towards zero as $r\to\infty$, which is accomplished by setting the first and last B-spline to $0$.
This reduces the matrices by an order of $2$, and thus we're left with a set of $n - k - 2$ equations.

\newparagraph
The matrices $\mathbf{B}$ and $\mathbf{H}$ are banded matrices of bandwidth $k - 1$, this is because the B-spline are uniquely $0$ after their span. 
Hence, we have one free parameter, $l$, which is the angular momenta.


\section{Result \& Discussion}\label{sec: result}
In this report, the following unit conversion is made, which result in the unit of energy being Hartree.
A \verb|c++| script was made to solve eq \eqref{eq: gen eq}, using eq \eqref{eq: B} -- \eqref{eq: H} and \eqref{eq: gauss-quad}.
The program determines the degree of Legendre polynomial depending on the spline order $k$, so for $k = 4$ the number degree of the Legendre polynomial would be $4$ as well.
This is beneficial when it comes to higher order polynomials, where the accuracy increases slightly.
The knot sequence was chosen to be equidistant, in the domain $[0, 25]$, a total of $50$ physical knots.
The reason for a linear knot sequence is that the B-splines were that the results seemed to be indifferent to the knot sequence. 

\newparagraph
In table \ref{tab: units}, some constants have been defined, which dominates the unit-analysis of the problem. Using these, the unit of energy is Hartree.

\begin{table}[H]
    \centering
    \caption{Unit conversion}
    \begin{tabular}{|c|c|c|}\hline
        $\hbar$ & $4\pi\epsilon_0$ & $m_e$\\\hline
        $1$ & $1$ & $1$\\\hline
    \end{tabular}
    \label{tab: units}
\end{table}\noindent
In eq \eqref{eq: gen eq} we solve for our reduced radial wave-function $P_{nl}(r)$, which differs slightly compared to the radial part of the wave-function.
Below in figure \ref{fig: different l} -- \ref{fig: l = 2}, the reduced radial part of the wave-function is presented for a variety of states.
The energies of the corresponding state were computed and is presented in the table below, table \ref{tab: energies}.
\begin{table}[H]
    \centering
    \caption{Energy of different states}
    \begin{tabular}{|c|c|}\hline
        State & Energy [Hartree] \\\hline
        $1s$ & $-0.499995$\\\hline
        $2s$ & $-0.124999$\\\hline
        $2p$ & $-0.125$ \\\hline
        $3s$ & $-0.0545922$\\\hline
        $3p$ & $-0.0549094$\\\hline
        $3d$ & $-0.0553215$\\\hline 
    \end{tabular}
    \label{tab: energies}
\end{table}\noindent

\begin{figure}[H]
    \centering
    \begin{subfigure}{0.45\textwidth}
        \includegraphics[scale = 0.8]{code/l0z1.pdf}
        \caption{Reduced radial part of $1s$ and $2s$}
        \label{fig: l = 0}
    \end{subfigure}
    \hfill
    \begin{subfigure}{0.45\textwidth}
        \includegraphics[scale = 0.8]{code/l1z1.pdf}
        \caption{Reduced radial part of $2s$ and $2p$ states}
        \label{fig: l = 1}
    \end{subfigure}
    \caption{Different states of hydrogen}
    \label{fig: different l}
\end{figure}\noindent
This result is comparable to the actual radial part, which is shown below for a set of states in fig \ref{fig: radial wave-function}
\begin{figure}[H]
    \centering
    \includegraphics[scale = 0.8]{code/l2z1.pdf}
    \caption{Reduced radial part of $3p$ and $3d$}
    \label{fig: l = 2}
\end{figure}\noindent
The wave-function, $\psi$, is by our assumption separable, into two parts, the radial and spherical parts, $R_{nl}(r)$ and $Y_{nl}(\theta, \phi)$.
This is allowed by symmetry, allows for easier visualization.
In the figures below, fig \ref{fig: radial wave-function}, the radial part of the wave-function $R_{nl}(r)$ is being shown.
Due to limitation in time, the spherical harmonics have not been implemented, although I really would like to.

\begin{figure}[H]
    \centering
    \begin{subfigure}{0.3\textwidth}
        \includegraphics[scale = 0.5]{code/l0z1r.pdf}
        \caption{Radial part of $\psi$, $1s$ \& $2s$}
    \end{subfigure}
    \hfill
    \begin{subfigure}{0.3\textwidth}
        \includegraphics[scale = 0.5]{code/l1z1r.pdf}
        \caption{Radial part of $\psi$, $2s$ \& $2p$}
    
    \end{subfigure}
    \hfill
    \begin{subfigure}{0.3\textwidth}
        \includegraphics[scale = 0.5]{code/l2z1r.pdf}
        \caption{Radial part of $\psi$, $3p$ \& $3d$}

    \end{subfigure}
    \caption{Radial wave-function of different states}
    \label{fig: radial wave-function}
\end{figure}\noindent
If the electric repulsion term in the Hamiltonian, eq \eqref{eq: hamiltonian}, is instead that of a uniform electric sphere,
\begin{align*}
    V(r) &= -\frac{Ze^2}{4\pi\epsilon_0} \frac{1}{2R_0}\left(3- \left(\frac{r}{R_0}\right)^2\right); \quad r\leq R_0\\
    &= -\frac{Ze^2}{4\pi\epsilon_0}\frac{1}{r}:\quad r> R_0,
\end{align*}then Hamiltonian loses its potential decay close to the core.
The energies of the $1s$ state is computed to be $E_0 = -0.361099$~Hartree, which is less than the previously found result.
The other states, loses the majority of the energy, which is expected, since they are located close to the core.
Furthermore, the radial part of the wave-function, gets clamped and is now contained in a smaller domain. This is seen in figure \ref{fig: mod pot}.
\begin{figure}[H]
    \centering
    \includegraphics[scale = 1]{code/l0z1vcv.pdf}
    \caption{Radial part of $\psi$, when treated as a uniform sphere}
    \label{fig: mod pot}
\end{figure}\noindent
At the point $R_0$ we have an interesting point, therefore it might be a good idea to point extra knots close to the point $R_0$.
This would allow for a more accurate representation close to $R_0$, since the B-splines would be able to capture the behavior of the wave-function better.
\newpage

\section{Conclusion}
In this report, we've successfully solved the hydrogen atom, by introducing a basis of polynomial.
This allowed us to solve the hydrogen atoms different orbitals, where the results agree with tabulated data.
The energies for the different states were found in solving the generalized eigen-value equation that arose in our basis choice, eq \eqref{eq: gen eq},
and we used Gaussian quadrature to compute the integrals in the equations \eqref{eq: B} -- \eqref{eq: H}.

\newparagraph
In doing this exercise, I found some usefulness of B-splines, and an appreciation in their simplicity but also their usefulness.
If I had more time, I would try to implement visualization of the spherical harmonics for the different state, to get a complete picture of the wave-function $\psi_{nl}(r,\theta, \phi)$.

%\printbibliography
\end{document}
 
